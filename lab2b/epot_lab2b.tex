\documentclass[a4paper,11pt]{article}

\usepackage[portuguese]{babel}
\usepackage[utf8]{inputenc}
\usepackage{amsmath}
\usepackage{graphicx}
\usepackage{hyperref}
\usepackage{float}
\usepackage{subfig}
\usepackage{fixltx2e}
\usepackage[bottom]{footmisc}
\usepackage{listings}
\usepackage{xargs}                      % Use more than one optional parameter in a new commands
\usepackage[pdftex,dvipsnames]{xcolor}  % Coloured text etc.
\usepackage[colorinlistoftodos,prependcaption,textsize=tiny]{todonotes}
\newcommandx{\unsure}[2][1=]{\todo[linecolor=red,backgroundcolor=red!25,bordercolor=red,#1]{#2}}
\newcommandx{\change}[2][1=]{\todo[linecolor=blue,backgroundcolor=blue!25,bordercolor=blue,#1]{#2}}
\newcommandx{\info}[2][1=]{\todo[linecolor=OliveGreen,backgroundcolor=OliveGreen!25,bordercolor=OliveGreen,#1]{#2}}
\newcommandx{\improvement}[2][1=]{\todo[linecolor=Plum,backgroundcolor=Plum!25,bordercolor=Plum,#1]{#2}}
\newcommandx{\thiswillnotshow}[2][1=]{\todo[disable,#1]{#2}}
\usepackage[font=footnotesize]{caption}
\usepackage[hypcap]{caption}
\usepackage[top=2.5cm, bottom=2.5cm, left=2.5cm, right=2.5cm]{geometry}
\usepackage{enumerate}
%\usepackage[siunitx,american]{circuitikz}

\setcounter{tocdepth}{3}
\setcounter{secnumdepth}{4}

\numberwithin{equation}{section}
\addto\captionsportuguese{\renewcommand{\contentsname}{Índice}}

\linespread{1.3}
\usepackage{indentfirst}

\begin{document}
\begin{titlepage}
\begin{center}

\hfill \break
\hfill \break

\includegraphics[width=0.3\textwidth]{img/logo}~\\[1cm] 

\textsc{\LARGE Instituto Superior Técnico}\\[0.25cm]
\textsc{\Large Mestrado Integrado em Engenharia Electrotécnica e de Computadores}\\[1.8cm]
\textsc{\huge Electrónica de Potência}\\[0.25cm]

\vspace{6mm}

{\huge \bfseries Conversor CA/CC Monofásico \linebreak Comandado de Onda Completa \\[0.7cm]}
{\bfseries Rectificador de onda completa totalmente comandado e semi-comandado \\[1cm]}

\begin{tabular}{ l l }
	João Bernardo Sequeira de Sá & \hspace{2mm} n.º 68254 \\
	Maria Margarida Dias dos Reis & \hspace{2mm} n.º 73099 \\
	Rafael Augusto Maleno Charrama Gonçalves & \hspace{2mm} n.º 73786 \\
	Nuno Miguel Rodrigues Machado & \hspace{2mm} n.º 74236
\end{tabular}

\vspace{7mm}

Turno de Segunda-feira das 17h00 - 20h00

\vfill

{\large Lisboa,  de Novembro de 2015} 
	
\end{center}
\end{titlepage}
	
\tableofcontents
\pagebreak

\section{Introdução}




\section{Condução do Trabalho}

\subsection{Retificador de onda completa totalmente comandado}

\subsubsection{Carga resistiva pura (R)}

\paragraph{Formas de onda da tensão e corrente na entrada}

\paragraph{Formas de onda da tensão e corrente na carga}

\paragraph{Formas de onda da tensão e corrente no tiristor}

\paragraph{Característica de comando do conversor}

\subsubsection{Carga indutiva RL}

\paragraph{Formas de onda da tensão e corrente na carga para funcionamento lacunar}

\todo{dizer porque razão a tensão na carga é negativa por algum tempo}

\todo{tensão medida na carga}

\paragraph{Formas de onda da tensão e corrente no tiristor}

\paragraph{Formas de onda da tensão e corrente para funcionamento não lacunar}

\paragraph{Característica de comando do conversor}

\subsection{Retificador de onda completa semi-comandado}

\subsubsection{Carga indutiva RL}

\paragraph{Formas de onda da tensão e corrente na entrada}

\paragraph{Formas de onda da tensão e corrente na carga}

\paragraph{Formas de onda da tensão e corrente no tiristor}

\paragraph{Característica de comando do conversor}

\todo{porque razão a corrente na carga nunca é negativa}

\todo{valor médio da tensão na carga para ângulo de diaspor deo 60º}

\todo{dizer se é possivel utilizar este circuito para controlar a velocidade de um motor CC com travagem regenerativa}

\todo{que tipo de filtro utilizaria para exigências de conteúdo harmónico. Pode ligar-se um condensador em paralelo na saída do retificador? porque?}


\pagebreak

\begin{thebibliography}{2}
	
	\bibitem{Kassakian}
	Kassakian, John G. et al (1992, June), Principles of Power Electronics, \textit{Addison-Wesley Publishing Company}

	\bibitem{Rashid}
	Rashid, Muahammad H. (2004), Power Electronics - Circuits, Devices and Applications, \textit{Prentice Hall}
	
	\bibitem{Silva}
	Silva, Fernando (1998), Eletrónica Industrial, Fundação Calouste Gulbenkian
	
\end{thebibliography}


\pagebreak



\end{document}
