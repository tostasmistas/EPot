\documentclass[a4paper,11pt]{article}

\usepackage[portuguese]{babel}
\usepackage[utf8]{inputenc}
\usepackage{amsmath}
\usepackage{graphicx}
\usepackage{hyperref}
\usepackage{float}
\usepackage{subfig}
\usepackage{fixltx2e}
\usepackage[bottom]{footmisc}
\usepackage{listings}
\usepackage{xargs}                      % Use more than one optional parameter in a new commands
\usepackage[pdftex,dvipsnames]{xcolor}  % Coloured text etc.
\usepackage[colorinlistoftodos,prependcaption,textsize=tiny]{todonotes}
\newcommandx{\unsure}[2][1=]{\todo[linecolor=red,backgroundcolor=red!25,bordercolor=red,#1]{#2}}
\newcommandx{\change}[2][1=]{\todo[linecolor=blue,backgroundcolor=blue!25,bordercolor=blue,#1]{#2}}
\newcommandx{\info}[2][1=]{\todo[linecolor=OliveGreen,backgroundcolor=OliveGreen!25,bordercolor=OliveGreen,#1]{#2}}
\newcommandx{\improvement}[2][1=]{\todo[linecolor=Plum,backgroundcolor=Plum!25,bordercolor=Plum,#1]{#2}}
\newcommandx{\thiswillnotshow}[2][1=]{\todo[disable,#1]{#2}}
\usepackage[font=footnotesize]{caption}
\usepackage[hypcap]{caption}
\usepackage[top=2.5cm, bottom=2.5cm, left=2.5cm, right=2.5cm]{geometry}
\usepackage{enumerate}
\usepackage[siunitx,american]{circuitikz}

\setcounter{tocdepth}{3}
\setcounter{secnumdepth}{4}

\numberwithin{equation}{section}
\addto\captionsportuguese{\renewcommand{\contentsname}{Índice}}

\linespread{1.3}
\usepackage{indentfirst}

\begin{document}
\begin{titlepage}
\begin{center}

\hfill \break
\hfill \break

\includegraphics[width=0.3\textwidth]{img/logo}~\\[1cm] 

\textsc{\LARGE Instituto Superior Técnico}\\[0.25cm]
\textsc{\Large Mestrado Integrado em Engenharia Electrotécnica e de Computadores}\\[1.8cm]
\textsc{\huge Electrónica de Potência}\\[0.25cm]

\vspace{6mm}

{\huge \bfseries Conversor CA/CC Monofásico \linebreak Comandado de Meia Onda \\[0.7cm]}
{\bfseries Rectificador de meia onda com carga R e RL \& com carga RL e díodo de roda \\[1cm]}

\begin{tabular}{ l l }
	João Bernardo Sequeira de Sá & \hspace{2mm} n.º 68254 \\
	Maria Margarida Dias dos Reis & \hspace{2mm} n.º 73099 \\
	Rafael Augusto Maleno Charrama Gonçalves & \hspace{2mm} n.º 73786 \\
	Nuno Miguel Rodrigues Machado & \hspace{2mm} n.º 74236
\end{tabular}

\vspace{7mm}

Turno de Segunda-feira das 17h00 - 20h00

\vfill

{\large Lisboa,  de Novembro de 2015} 
	
\end{center}
\end{titlepage}
	
\tableofcontents
\pagebreak

\section{Introdução}

Com este trabalho pretende estudar-se o funcionamento do conversor CA/CC (Retificador) monofásico comandado de meia onda.

Este tipo de retificadores tira o seu nome da capacidade que o Tiristor tem em controlar a sua tensão de saída através do ângulo de disparo. O Tiristor é levado à condução pela aplicação de um sinal na textit{Gate} e passa ao corte através de comutação natural na maioria dos casos, ou comutação forçada no caso de circuitos fortemente indutivos.

Pode dizer-se que este trabalho está dividido em duas partes. Na primeira estuda-se o retificador de meia onda simples com carga R e RL. Na segunda parte trata-se da inclusão de um díodo de roda livre e estuda-se o comportamento deste novo circuito de potência com uma carga RL.

O comportamento do Retificador de meia onda com uma carga resistiva pura é tal que a corrente e a tensão na carga têm que apresentar a mesma polaridade em qualquer altura. Devido ao comportamento do Tiristor apenas existe tensão na saída durante alternâncias positivas, pois quando a corrente se anula no Tiristor este passa ao corte, sendo esta característica do comportamento que leva ao nome "de meia onda".

Já no caso de uma carga RL existe uma ligeira desfasagem entre tensão e corrente, devido à presença da bobine, pelo que a corrente no Tiristor apenas se anula quando a tensão já se encontra na sua alternância negativa. Embora continuemos a ter apenas meia onda, já não existe uma distinção certa entre conduzir apenas durante uma das alternâncias.

Na segunda parte deste trabalho inclui-se ao circuito um Díodo de Roda Livre. O Díodo de Roda Livre é também conhecido como Díodo \textit{Snubber} ou Díodo \textit{Flyback}. Este é utilizado para prevenir valores negativos de tensão na saída assim como eliminar os picos de tensão quando se têm cargas indutivas a sofrer comutação.



\section{Condução do Trabalho}

\subsection{Estudo do circuito de disparo}

O circuito de disparo utilizado neste trabalho pode ser representado por um diagrama de blocos tal como se observa na \autoref{fig:circuit_1}.

Como este está preparado para servir como circuito de disparo para um Retificador de Ponte Completa, inclui a capacidade para realizar o \textit{Trigger} de 4 Tiristores, sendo estes disparados aos pares e consecutivamente. O circuito é constituído por um transformador que amostra a tensão de entrada (TR\_SINC), garantindo a sincronia entre a geração dos impulsos e a tensão; um circuito integrado (UAA145), que detecta a passagem da tensão da rede por zero e gera impulsos em $\alpha$ e $\alpha + \pi$ e um circuito integrado (NE555) e duas portas lógicas $NAND$, que combinam os impulsos num trem e garantem que a potência da \textit{Gate} dos Tiristores não é excedida.

\begin{figure}[h]
	\centering
	\includegraphics[keepaspectratio=true, scale=0.8]{teoricas/circuito_disparo.jpg}
	\caption{Diagrama de blocos do Circuito de Disparo.}
	\label{fig:circuit_1}
	\vspace{-0.8em}
\end{figure}

\subsubsection{Formas de onda dos sinais de disparo}

Para que fosse possível estudar o funcionamento do circuito de disparo começou por se observar as formas de onda da tensão amostrada de entrada e dos impulsos gerados.

\todo{inserir imagem AB 1}

\todo{inserir imagem AC 1}

Na \textcolor{red}{Figura AB 1} pode ver-se a amarelo a tensão amostrada pelo transformador e a azul o impulso de \textit{Trigger} gerado para $\alpha$. Já na \textcolor{red}{Figura AC 1} pode observar-se a amarelo novamente a tensão amostrada e a azul o impulso de \textit{Trigger} gerado para $\alpha + \pi$.

A posição destes dois impulsos pode ser controlada através do valor do ângulo de disparo $\alpha$ rodando um manipulo presente na placa do Circuito de Disparo. Ao mudar este valor, as formas de onda são observáveis, tanto para a tensão amostrada e o impulso $\alpha$ como para a mesma tensão e o impulso  $\alpha + \pi$, na \textcolor{red}{Figura AB 2} e \textcolor{red}{Figura AC 2} respectivamente.

\todo{inserir imagem AB 2}

\todo{inserir imagem AC 2}


\subsubsection{Trem de impulsos}

Como a frequência de operação é a frequência da rede, $50$ Hz, os impulsos de disparo devem ter durações de poucos milissegundos, evitando-se efeitos negativos de \textit{latching}, sendo estes especialmente prevalentes no caso de cargas indutivas em que o ângulo de disparo será elevado. No entanto estas durações podem saturar o transformador de impulsos ou provocar perdas elevadas na \textit{Gate} do Tiristor.

Para resolver ambos estes problemas utiliza-se um trem de impulsos de \textit{Trigger}.

\todo{inserir imagem trem de impulsos e A}

Colocou-se entre a \textit{Gate} e o Cátodo uma resistência de $33$ $\Omega$ para simular a \textit{Gate} do tiristor. Na \textcolor{red}{Figura trem de impulsos e A} pode assim ver-se a azul a queda de tensão nessa resistência e a amarelo o trem de impulsos.

\todo{inserir imagem trem de impulsos e B}

Na \textcolor{red}{Figura trem de impulsos e B} pode também observar-se o trem de impulsos, a amarelo, face ao impulso original correspondente, a azul.


\subsection{Estudo do circuito de potência}

\subsubsection{Carga resistiva pura (R)}

Para que se possa estudar o funcionamento do Retificador para cargas Resistivas puras utilizou-se um reóstato como carga, regulando-o para perto de meio ficando-se com cerca de $16$ $\Omega$.

\paragraph{Formas de onda da tensão e corrente na entrada}

\todo{inserir imagem Carga Resistiva}

Na \textcolor{red}{Figura Carga Resistiva} observa-se o funcionamento do circuito quando sujeito uma carga resistiva pura. A tensão de entrada pode ser vista a amarelo estando a tensão e corrente na carga a azul e roxo respetivamente.

\paragraph{Formas de onda da tensão e corrente na saída}


\unsure{faltam bastantes coisas aqui. temos que decidir como se faz}

\paragraph{Formas de onda da tensão e corrente no tiristor}


\paragraph{Característica de comando do conversor}


\subsubsection{Carga indutiva RL}

\paragraph{Formas de onda da tensão e corrente na saída}

\todo{5.2g}

\paragraph{Formas de onda da tensão e corrente no tiristor}

\todo{5.2h}

\unsure{fizemos a 5.2i e a 5.2j?}

\subsubsection{Carga indutiva RL e díodo de roda livre}

\paragraph{Formas de onda da tensão e corrente na saída}

\todo{5.2l}

\paragraph{Formas de onda da tensão e corrente no tiristor}

\todo{5.2m}

\paragraph{Formas de onda da tensão e corrente no díodo e na bobina}

\todo{5.2n}

\section{Simulação do Trabalho de Laboratório}

\todo{this is our expertise teddy}

\subsection{Circuito de disparo}

\subsection{Circuito de potência}

\section{Referências}

Kassakian, John G. et al (1992, June). Principles of Power Electronics. \textit{Addison-Wesley Publishing Company}

Rashid, Muahammad H. (2004). Power Electronics - Circuits, Devices and Applications. \textit{Prentice Hall}

\pagebreak

\end{document}
