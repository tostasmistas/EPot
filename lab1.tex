\documentclass[a4paper,11pt]{article}

\usepackage{tecnico_relatorio}
\usepackage[hypcap]{caption}
\usepackage[top=2.5cm, bottom=2.5cm, left=2.5cm, right=2.5cm]{geometry}
\usepackage{amsmath}
\usepackage{enumerate}
\usepackage[siunitx,american]{circuitikz}
\usepackage{listings}

\begin{document}
	\trSetImage{img/tecnico_logo}{6cm} % Logotipo do Técnico
	\trSetSubject{Electrónica de Potência}
	\trSetType{Laboratório 1}
	\trSetTitle{Circuito de Disparo de um Tiristor
	
	Circuito com Carga Ressonante}

	
	\trSetBoxStyle{0.2}
	
	\trSetAuthorNr{4}
	
	\trSetAuthors
	{
		\begin{center}
			João de Sá

			68254
		\end{center}
	}{
		\begin{center}
			Maria Margarida Reis

			73099
		\end{center}
	}{
		\begin{center}
			Rafael Gonçalves

			73786
		\end{center}
	}{
		\begin{center}
			Nuno Machado

			74236
		\end{center}
	}
		
	\trSetProfessor{Prof. Beatriz Borges}
	
	\trMakeCover
	
	\tableofcontents
	\pagebreak
	
	\section{Introdução}
	
	Pretende-se com este trabalho estudar o comportamento do tirístor, com especial interesse na passagem à condução e ao corte deste dispositivo, assim como evidenciar alguns aspetos da sua utilização em circuitos de conversão de potência.
	
	O tirístor, ou Retificador Controlado de Silício, é o dispositivo indicado para comandar tensões e correntes de valor elevado, sendo capaz de suportar potências da ordem dos $10$ MW. É composto por três terminais, o elétrodo de disparo, ou “Gate” (G), ânodo (A) e cátodo (K). Através da Gate pode levar-se o dispositivo à condução, caso este esteja polarizado diretamente nos terminais de ânodo e cátodo, através de um impulso. Por norma os terminais de potência, ânodo e cátodo, desempenham funções semelhantes aos terminais do díodo. Em oposição ao transístor, o tirístor é um dispositivo que possui memória; uma vez que seja colocado à condução não regressa ao estado de bloqueio através de atuação na gate, mas sim através de um anulamento da corrente, polarização inversa, comportamento idêntico ao do díodo. Gera-se assim uma necessidade para que, caso o circuito em que o dispositivo é aplicado não possua uma comutação natural, se recorra a técnicas de comutação forçada.
	
	Estas técnicas de comutação forçada são concebidas normalmente com recurso a componentes reativos, como sejam a bobine ou o condensador, para que possa ser estabelecida uma polarização inversa aos terminais do tirístor num certo período de tempo do funcionamento do circuito. Estas técnicas levam no entanto a perdas, pelo que as frequências de operação sejam da ordem de $500$ a $1.5$ kHz.
	
	Atualmente existe tendência para usar como alternativa IGBT’s ou GTO’s.
	
	\section{Circuito de Disparo}
	
	De forma a estudar o comportamento de circuitos com semicondutores de potência é necessário, em primeira instância, realizar o circuito de “drive” ou ataque ao terminal de controlo, ou no caso de tirístores o circuito de disparo. Este circuito tem a função de estabelecer o sinal de comando do tirístor, sendo este aplicado entre a Gate e o cátodo, assim como estabelecer o isolamento galvânico entre o circuito de potência e o circuito de controlo. Pode observar-se este circuito na \autoref{fig:circuit_1}.
	
	\begin{figure}[h]
		\centering
		\includegraphics[width=\linewidth]{img/trigger_circuit}
		\caption{Circuito de Disparo}
		\label{fig:circuit_1}
	\end{figure}
	
	O objetivo neste trabalho é assim realizar este circuito com uma frequência de $1$ kHz fazendo para isso uso de um sinal com esta frequência originado por um Gerador de Impulsos (GI). O circuito de disparo será então composto por uma monoestável que reage ao flanco ascendente do sinal originado pelo GI; tem-se assim à saída da monoestável um impulso cuja duração será função da resistência R e condensador C. A duração deste impulso deve ser definida consoante as características da Gate do tirístor que se está a utilizar, sendo neste caso de $10$ $\mu$s. Este impulso tem no entanto que ser amplificado para que seja injetada corrente suficiente na Gate do tirístor. Usa-se assim um transístor de ganho elevado transitando da saturação ao corte, estabelecendo uma tensão no primário do transformador, sempre que surja o impulso na saída da monoestável. As formas de onda destes impulsos podem ser observadas na \autoref{fig:circuit_2}.
	
	\begin{figure}[h]
		\centering
		\includegraphics[width=\linewidth]{img/trigger_waveform}
		\caption{Formas de onda das tensões no circuito de disparo}
		\label{fig:circuit_2}
	\end{figure}
	
	O transformador serve também para que se obtenha o isolamento galvânico entre os circuitos de disparo e potência.

	\section{Montagem e equipamento}

	A montagem presente na placa impressa utilizada no laboratório pode ser observada na \autoref{fig:circuit_3}.
	
	\begin{figure}[h]
		\centering
		\includegraphics[width=\linewidth]{img/assembly_circuit}
		\caption{Esquema elétrico do circuito de disparo presente na placa impressa}
		\label{fig:circuit_3}
	\end{figure}
	
	Tal como dito na secção acima, a duração do impulso será definida por R e C segundo a seguinte fórmula dada pelo fabricante:
	
	$$ T = 2.88 \, R \, C $$
	
	Para que se tenha $10$ $\mu$s faz-se assim uso de uma resistência com $10$ k$\Omega$ e $0.4$ nF, sem necessidade de uma grande precisão nos valores pois a exatidão do tempo de disparo neste circuito não é prevalente.
	
	O equipamento a utilizar na condução do trabalho é assim:
	
	\begin{itemize}
	\item 1 Osciloscópio;
	\item 1 Sonda de corrente;
	\item 1 Gerador de impulsos;
	\item 2 Fontes de alimentação;
	\item 2 Multímetros;
	\item 1 Placa de circuito impresso;
	\end{itemize}
	
	\section{Condução do Trabalho}
\end{document}
